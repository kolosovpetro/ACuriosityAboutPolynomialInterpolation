\documentclass[12pt,letterpaper,oneside,reqno]{amsart}
\usepackage{amsfonts}
\usepackage{amsmath}
\usepackage{amssymb}
\usepackage{amsthm}
\usepackage{float}
\usepackage{mathrsfs}
\usepackage{colonequals}
\usepackage[font=small,labelfont=bf]{caption}
\usepackage[left=1in,right=1in,bottom=1in,top=1in]{geometry}
\usepackage[pdfpagelabels,hyperindex,colorlinks=true,linkcolor=blue,urlcolor=magenta,citecolor=green]{hyperref}
\usepackage{graphicx}
\linespread{1.7}
\emergencystretch=1em
\usepackage{array}
\usepackage{etoolbox}
\apptocmd{\sloppy}{\hbadness 10000\relax}{}{}
\raggedbottom

\newcommand \bernoulli [2][B] {{#1}\sb{#2}}
\newcommand \coeffA [3][A] {{\mathbf{#1}} \sb{#2,#3}}
\newcommand \polynomialP [4][P]{{\mathbf{#1}}\sp{#2} \sb{#3}(#4)}

% free foot note
\let\svthefootnote\thefootnote
\newcommand\freefootnote[1]{%
    \let\thefootnote\relax%
    \footnotetext{#1}%
    \let\thefootnote\svthefootnote%
}


\newtheorem{theorem}{Theorem}[section]
\newtheorem{corollary}[theorem]{Corollary}
\newtheorem{lemma}[theorem]{Lemma}
\newtheorem{example}[theorem]{Example}
\newtheorem{conjecture}[theorem]{Conjecture}
\newtheorem{definition}[theorem]{Definition}
\newtheorem{question}[theorem]{Question}

%\numberwithin{equation}{section}

\title[A curiosity about Polynomial Interpolation]
{A curiosity about Polynomial Interpolation}
\author[Petro Kolosov]{Petro Kolosov}
%\address{Software Developer, DevOps Engineer}
%\email{kolosovp94@gmail.com}
%\urladdr{https://kolosovpetro.github.io}
%\keywords{
%    Keyword1, Keyword2
%}
%\subjclass[2010]{26E70, 05A30}
%\date{\today}
\hypersetup{
    pdftitle={A curiosity about Polynomial Interpolation},
    pdfsubject={
        Your Subject List
    },
    pdfauthor={Petro Kolosov},
    pdfkeywords={
        Your Keywords list
    }
}
\begin{document}
    \begin{abstract}
        Interpolation of cubes expected to be
\[
    n^3 = 6 \binom{n}{3} + 6 \binom{n}{2} + \binom{n}{1} + 0 \binom{n}{0}
\]
but got
\[
    n^3 = \sum_{r=0}^{m} \sum_{k=1}^{n} \mathbf{A}_{m,r} k^r (n-k)^r
\]

    \end{abstract}

    \maketitle

    \tableofcontents

%    \freefootnote{Sources: \url{https://github.com/kolosovpetro/ACuriosityAboutPolynomialInterpolation}}

    \section{Introduction} \label{sec:introduction}
    Back then, in 2016 being a student at the faculty of mechanical engineering,
I remember myself playing with finite differences of the polynomial $n^3$ over the domain of natural numbers $n\in\mathbb{N}$
having at most $0 \leq n \leq 20$ values.
Looking to the values in my finite difference tables, the first and very naive question that came to my mind was
\begin{question}
    Is it possible to re-assemble the value of the polynomial $n^3$ backwards
    having its finite differences?
\end{question}
The answer to this question is definitely \textit{Yes}, by utilizing certain interpolation principles.
Interpolation is a process of finding new data points based on the range of a discrete set of known data points.
Interpolation has been well-developed in between 1674--1684
by Issac Newton's fundamental works, nowadays known as foundation of classical interpolation
theory~\cite{meijering2002chronology}.

At that time, in 2016, I was a first-year mechanical engineering undergraduate,
so that due to lack of knowledge and perspective of view I started re-inventing interpolation
formula myself, fueled by purest passion and feeling of mystery.
\textit{All the mathematical laws and relations exist from the very beginning, we only reveal and describe them},
I thought.
That mindset truly inspired me, so my own mathematical journey has been started.
Let's begin considering the table of finite differences of integer cubes $n^3$
\input{sections/figures/01_fig_finite_differences_cubes}


    \section{Generalizations}\label{sec:generalizations}
    Assume that our previously obtained identities
$n^3 = \sum_{k=0}^{n-1} 6k(n-k) + 1$ and $n^3 = \sum_{k=1}^{n} 6k(n-k) + 1$
have explicit form as follows
\begin{align*}
    n^3 = \sum_{k} \coeffA{1}{1} k^1(n-k)^1 + \coeffA{1}{0} k^0(n-k)^0
\end{align*}
where $\coeffA{1}{1} = 6$ and $\coeffA{1}{0} = 1$, respectively.
Therefore, let be a conjecture
\begin{conjecture}
    \label{conj:generalization}
    For every $n\geq 1, \; n,m\in\mathbb{N}$ there are coefficients $\coeffA{m}{0}, \coeffA{m}{1}, \ldots, \coeffA{m}{m}$ such that
    \begin{align*}
        n^{2m+1} = \sum_{k=1}^{n} \coeffA{m}{0} k^0 (n-k)^0 + \coeffA{m}{1} (n-k)^1
        + \cdots + \coeffA{m}{m} k^m (n-k)^m
    \end{align*}
\end{conjecture}
Note that conjecture above assumes the convention $0^0=1$,
reader may found a comprehensive discussion of it in~\cite{knuth1992two}.

Long story short, above conjecture is true, so that real coefficients $\coeffA{m}{r}$ are following
\input{sections/figures/05_fig_coefficients_a}
These coefficients $\coeffA{m}{r}$ are defined via a recurrence relation involving Binomial coefficients
and Bernoulli numbers
\begin{definition} (Definition of coefficient $\coeffA{m}{r}$.)
    \begin{align*}
        \label{eq:definition_coefficient_a}
        \coeffA{m}{r} =
        \begin{cases}
        (2r+1)
            \binom{2r}{r} & \mathrm{if} \; r=m \\
            (2r+1) \binom{2r}{r} \sum_{d \geq 2r+1}^{m} \coeffA{m}{d} \binom{d}{2r+1} \frac{(-1)^{d-1}}{d-r}
            \bernoulli{2d-2r} & \mathrm{if} \; 0 \leq r<m \\
            0 & \mathrm{if} \; r<0 \; \mathrm{or} \; r>m
        \end{cases}
    \end{align*}
\end{definition}
where $\bernoulli{t}$ are Bernoulli numbers~\cite{bateman1953higher}.
It is assumed that $\bernoulli{1}=\frac{1}{2}$.
Properties of the coefficients $\coeffA{m}{r}$
\begin{itemize}
    \item $\coeffA{m}{m} = (2m+1) \binom{2m}{m}$
    \item $\coeffA{m}{r} = 0$ for $m < 0$ and $r > m$
    \item $\coeffA{m}{r} = 0$ for $r < 0$
    \item $\coeffA{m}{r} = 0$ for $\frac{m}{2} \leq r < m$
    \item $\coeffA{m}{0} = 1$ for $m \geq 0$
    \item $\coeffA{m}{r}$ are integers for $m \leq 11$
    \item Row sums: $\sum_{r=0}^{m} \coeffA{m}{r} = 2^{2m+1} - 1$
\end{itemize}

Proof of conjecture~\eqref{conj:generalization} as well as other discussions on topics above can be found
in literature~\cite{alekseyev2018mathoverflow, kolosov2024history, kolosov2016link, kolosov2022_an_unusual_identity, kolosov2023polynomial}.
Few OEIS sequences were contributed as well~\cite{oeis_kolosov2017third, oeis_kolosov2018fifth,
    oeis_kolosov2018coefficientspolynomial3, oeis_kolosov2018coefficientspolynomial2,
    oeis_kolosov2018coefficientspolynomial1}.

Very well, let's wrap up this technical section and move on to the more engaging discussions.


    \section{Discussions}\label{sec:discussion}
    \subsection{Interpolation}\label{subsec:interpolation}
Current manuscript starts from certain interpolation technique shown on base case of cubes, where the key
identity is the tricky rearrangement terms of the sum $\sum_k \Delta k^3$ in~\eqref{eq:rearranged-cubes}.
This rearrangement was done instead of applying Faulhaber's formula on $\Delta n^3 = 1 + 6\sum_{k=1}^{n} k$ which
leads to well-known result involving Binomial theorem:
$(n+1)^3 - n^3 = 1 + 6\sum_{k=1}^{n} k = 1 + 6 \frac{1}{2}(n+n^2) = 1 + 3n^3 + 3n$.


%    \section{Conclusions}\label{sec:conclusions}
%    \input{sections/03_conclusions}

    \bibliographystyle{unsrt}
    \bibliography{ACuriosityAboutPolynomialInterpolation}
    \noindent \textbf{Version:} \input{sections/version}

\end{document}
