\documentclass[12pt,letterpaper,oneside,reqno]{amsart}
\usepackage{amsfonts}
\usepackage{amsmath}
\usepackage{amssymb}
\usepackage{amsthm}
\usepackage{float}
\usepackage{mathrsfs}
\usepackage{colonequals}
\usepackage[font=small,labelfont=bf]{caption}
\usepackage[left=1in,right=1in,bottom=1in,top=1in]{geometry}
\usepackage[pdfpagelabels,hyperindex,colorlinks=true,linkcolor=blue,urlcolor=magenta,citecolor=green]{hyperref}
\usepackage{graphicx}
\linespread{1.7}
\emergencystretch=1em
\usepackage{array}
\usepackage{etoolbox}
\apptocmd{\sloppy}{\hbadness 10000\relax}{}{}
\raggedbottom

\newcommand \bernoulli [2][B] {{#1}\sb{#2}}
\newcommand \coeffA [3][A] {{\mathbf{#1}} \sb{#2,#3}}
\newcommand \polynomialP [4][P]{{\mathbf{#1}}\sp{#2} \sb{#3}(#4)}

% free foot note
\let\svthefootnote\thefootnote
\newcommand\freefootnote[1]{%
    \let\thefootnote\relax%
    \footnotetext{#1}%
    \let\thefootnote\svthefootnote%
}


\newtheorem{theorem}{Theorem}[section]
\newtheorem{corollary}[theorem]{Corollary}
\newtheorem{lemma}[theorem]{Lemma}
\newtheorem{example}[theorem]{Example}
\newtheorem{conjecture}[theorem]{Conjecture}
\newtheorem{definition}[theorem]{Definition}
\newtheorem{question}[theorem]{Question}

%\numberwithin{equation}{section}

\title[A curiosity about Polynomial Interpolation]
{A curiosity about Polynomial Interpolation}
\author[Petro Kolosov]{Petro Kolosov}
%\address{Software Developer, DevOps Engineer}
%\email{kolosovp94@gmail.com}
%\urladdr{https://kolosovpetro.github.io}
%\keywords{
%    Keyword1, Keyword2
%}
%\subjclass[2010]{26E70, 05A30}
%\date{\today}
\hypersetup{
    pdftitle={A curiosity about Polynomial Interpolation},
    pdfsubject={
        Your Subject List
    },
    pdfauthor={Petro Kolosov},
    pdfkeywords={
        Your Keywords list
    }
}
\begin{document}
    \begin{abstract}
        Interpolation of cubes expected to be
\[
    n^3 = 6 \binom{n}{3} + 6 \binom{n}{2} + \binom{n}{1} + 0 \binom{n}{0}
\]
but got
\[
    n^3 = \sum_{r=0}^{m} \sum_{k=1}^{n} \mathbf{A}_{m,r} k^r (n-k)^r
\]

    \end{abstract}

    \maketitle

    \tableofcontents

%    \freefootnote{Sources: \url{https://github.com/kolosovpetro/ACuriosityAboutPolynomialInterpolation}}

    \section{Introduction} \label{sec:introduction}
    Interpolation is a process of finding new data points based on the range of a discrete set of known data points.
Interpolation has been well-developed in between 1674--1684
by Issac Newton's fundamental works, nowadays known as foundation of classical interpolation
theory~\cite{meijering2002chronology}.

The first time I found interpolation interesting was in 2016 when I observed a table of finite differences of cubes.
Back then, I was a first-year mechanical engineering undergraduate.
Due to my lack of mathematical knowledge, I started re-inventing interpolation
formulas myself, fueled by pure passion and a sense of mystery.
\textit{All the mathematical laws and relations exist from the very beginning; we only reveal and describe them},
I thought.
That mindset truly inspired me, and thus, my own mathematical journey began.

Consider finite differences of cubes $n^3$
\begin{table}[H]
    \begin{center}
        \setlength\extrarowheight{-6pt}
        \begin{tabular}{c|cccc}
            $n$ & $n^3$ & $\Delta(n^3)$ & $\Delta^2(n^3)$ & $\Delta^3(n^3)$ \\
            \hline
            0   & 0     & 1             & 6               & 6               \\
            1   & 1     & 7             & 12              & 6               \\
            2   & 8     & 19            & 18              & 6               \\
            3   & 27    & 37            & 24              & 6               \\
            4   & 64    & 61            & 30              & 6               \\
            5   & 125   & 91            & 36              &                 \\
            6   & 216   & 127           &                 &                 \\
            7   & 343   &               &                 &
        \end{tabular}
    \end{center}
    \caption{Table of finite differences of the polynomial $n^3$.} \label{tab:table}
\end{table}

The problem of interpolation of polynomials is a classical problem in mathematics and has been widely studied in literature.
For instance, Concrete mathematics~\cite[p. 190]{graham1994concrete} gives interpolation of cubes by using
Newton's interpolation formula
\[
    n^3 = 6 \binom{n}{3} + 6 \binom{n}{2} + \binom{n}{1} + 0 \binom{n}{0}
\]
because
\begin{align*}
    f(x) = \Delta^{d} f(0) \binom{x}{d} +  \Delta^{d-1} f(0) \binom{x}{d-1} + \cdots + f(0) \binom{x}{0}
    = \sum_{r=0}^{d} \Delta^{d-r} f(0) \binom{x}{d-r}
\end{align*}
However, interpolation of cubes can be also done in a different way.
The key point that interpolation formula above iterates over the order $d$ of finite difference.
Alternatively, we can interpolate cubes $n^3$ as a sum of first order finite difference $\Delta$ as follows
\begin{align*}
    n^3 = \Delta 0^3 + \Delta 1^3 + \Delta 2^3 + \cdots + \Delta (n-1)^3 = \sum_{k=0}^{n-1} \Delta k^3
\end{align*}
We know that $\Delta^3 n^3 = 6$ is the constant for each $n$.
The second difference of cubes $\Delta^2 n^3$ is a linear relation in terms of third order finite difference
$\Delta^3 n^3$
\begin{align}
    \label{eq:second-difference-in-terms-of-third}
    \Delta^2 n^3 = (n+1) \Delta^3 n^3 = 6(n+1)
\end{align}
Finally, the first order finite difference $\Delta n^3$ is the following relation in terms of second
order finite difference
\begin{align*}
    \Delta n^3 = \Delta 0^3 + \Delta^2 0^3 + \Delta^2 1^3 + \cdots + \Delta^2 (n-1)^3 = 1 + \sum_{k=0}^{n-1} 6(k+1)
\end{align*}
Altering summation bounds yields
\begin{align}
    \label{eq:first-difference-in-terms-of-second}
    \Delta n^3 = 1+6\cdot0+6\cdot1+6\cdot2+6\cdot3+\cdots+6\cdot n = 1 + 6 \sum_{k=0}^{n} k
\end{align}
Therefore, we are able to express first order finite difference of cubes in form of sums as follows
\begin{align*}
    \Delta(0^3) &= 1+6 \cdot 0 \\
    \Delta(1^3) &= 1+6\cdot0+6\cdot1 \\
    \Delta(2^3) &= 1+6\cdot0+6\cdot1+6\cdot2 \\
    \Delta(3^3) &= 1+6\cdot0+6\cdot1+6\cdot2+6\cdot3
\end{align*}
Now it is time to assemble all the results above to get the polynomial $n^3$.
Having the relation in cubes $n^3 = \Delta 0^3 + \Delta 1^3 + \Delta 2^3 + \cdots + \Delta (n-1)^3$
we get
\begin{align}
    \label{eq:cubes-as-sum-of-first-order-differences}
    n^3 &= [1+6\cdot0]+[1+6\cdot0+6\cdot1]+[1+6\cdot0+6\cdot1+6\cdot2] \nonumber \\
    &+ \cdots + [1+6\cdot0+6\cdot1+6\cdot2+\cdots+6\cdot(n-1)]
\end{align}
By rearranging the terms of the equation above, we get summation in terms of $k (n-k)$
\begin{align}
    \label{eq:rearranged-cubes}
    n^3 = n &+ [(n-0) \cdot 6 \cdot 0] + [(n-1)\cdot6\cdot1] + [(n-2)\cdot6\cdot2] \nonumber \\
    &+ \cdots + [(n-k)\cdot 6 \cdot k] + \cdots + [1\cdot6\cdot(n-1)]
\end{align}
By applying compact sigma sum notation yields an identity for cubes $n^3$
\begin{align}
    \label{eq:apply-sigma-notation}
    n^3 = n + \sum_{k=0}^{n-1} 6k(n-k)
\end{align}
The term $n$ in the sum above can be moved under sigma notation, because there is exactly $n$ iterations, therefore
\begin{align}
    \label{eq:move-n-under-sigma}
    n^3 = \sum_{k=0}^{n-1} 6k(n-k) + 1
\end{align}
By inspecting the expression $6k(n-k) + 1$ we iterate under summation,
we can notice that it is symmetric over $k$, let be $T(n,k) = 6k(n-k) + 1$, then
\begin{align}
    \label{eq:symmetry}
    T(n,k) = T(n,n-k)
\end{align}
This symmetry allows us to alter summation bounds again, so that
\begin{align}
    \label{eq:alter-summation-bounds}
    n^3 = \sum_{k=1}^{n} 6k(n-k) + 1
\end{align}
Curiously enough that although $\sum_{k=0}^{n-1} 6k(n-k) + 1$ and $\sum_{k=1}^{n} 6k(n-k) + 1$ both simplify to $n^3$,
they produce different closed forms.
Let be $P(n, q) = \sum_{k=0}^{q-1} 6k(n-k) + 1$ and $Q(n, q) = \sum_{k=1}^{q} 6k(n-k) + 1$, then
\begin{align*}
    P(n, q) =
    \begin{cases}
        q = 1: & 1 \\
        q = 2: & -4 + 6 n \\
        q = 3: & -27 + 18 n
    \end{cases}
\end{align*}
\begin{align*}
    Q(n, q) =
    \begin{cases}
        q = 1: & -5 + 6 n \\
        q = 2: & -28 + 18 n \\
        q = 3: & -81 + 36 n
    \end{cases}
\end{align*}
Now let's take a breath and briefly summarize all the results we got so far.
So, we have successfully interpolated the polynomial $n^3$ using discrete set of finite differences data points
applying the following algorithm
\begin{enumerate}
    \item Express second finite difference as linear relation in terms of third
    finite difference~\eqref{eq:second-difference-in-terms-of-third}
    \item Express first finite difference in terms of second~\eqref{eq:first-difference-in-terms-of-second}
    \item Express cubes as sum of first order finite differences in $n$~\eqref{eq:cubes-as-sum-of-first-order-differences}
    \item Rearrange the terms of the sum~\eqref{eq:rearranged-cubes}
    \item Apply sigma notation~\eqref{eq:apply-sigma-notation}
    \item Move $n$ under sigma~\eqref{eq:move-n-under-sigma}
    \item Apply symmetry~\eqref{eq:symmetry}
    \item Alter summation bounds~\eqref{eq:alter-summation-bounds}
\end{enumerate}
\label{itm:cubes-algorithm}




%    \section{Conclusions}\label{sec:conclusions}
%    Conclusions of your manuscript.

    \bibliographystyle{unsrt}
    \bibliography{ACuriosityAboutPolynomialInterpolation}
    \noindent \textbf{Version:} \texttt{Local-0.1.0}


\end{document}
