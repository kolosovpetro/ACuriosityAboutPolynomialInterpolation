\subsection{Interpolation}\label{subsec:interpolation}
Current manuscript starts from certain polynomial technique shown on base case of cubes, where the key
identity is the tricky rearrangement of terms
in the sum $n^3=\sum_{k=0}^{n-1} \Delta k^3$ in~\eqref{eq:rearranged-cubes}.
This rearrangement was done instead of applying Faulhaber's formula on $\Delta n^3 = 1 + 6\sum_{k=1}^{n} k$ which
leads to well-known result involving Binomial theorem:
$\Delta n^3 = 1 + 6\sum_{k=1}^{n} k = 1 + 6 \frac{1}{2}(n+n^2) = 1 + 3n^3 + 3n$.
%The main question now is: Can we consider the process of finding identities for cubes
%~\eqref{eq:move-n-under-sigma},~\eqref{eq:alter-summation-bounds} as an interpolation method?
In context of rearrangement~\eqref{eq:rearranged-cubes} and moving $n$ under summation~\eqref{eq:move-n-under-sigma}
\begin{question}
    \label{question:interpolation_odd_power}
    Can we consider the process of finding identities for cubes
    ~\eqref{eq:move-n-under-sigma},~\eqref{eq:alter-summation-bounds} as an interpolation method?
\end{question}

The steps~\eqref{eq:rearranged-cubes} and~\eqref{eq:move-n-under-sigma} is the only distinct from well-known
result
\begin{align*}
    n^3 = \sum_{k=0}^{n-1} \sum_{r=0}^{2} \binom{3}{r} k^r
\end{align*}
Instead, we arrived to identities
\begin{align*}
    n^3 = \sum_{k=0}^{n-1} 6k(n-k) + 1; \quad \quad n^3 = \sum_{k=1}^{n} 6k(n-k) + 1
\end{align*}
\begin{question}
    \label{question:interpolation_generalization}
    Assuming that question~\eqref{question:interpolation_odd_power} is true,
    can we consider the conjecture~\eqref{conj:generalization} as an interpolation method for odd powers?
\end{question}
