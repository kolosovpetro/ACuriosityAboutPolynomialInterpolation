%Back then, in 2016 being a student at the faculty of mechanical engineering,
%I remember myself playing with finite differences of the polynomial $n^3$ over the domain of natural numbers $n\in\mathbb{N}$
%having at most $0 \leq n \leq 20$ values.
%Looking to the values in my finite difference tables, the first and very naive question that came to my mind was
%\begin{question}
%    Is it possible to re-assemble the value of the polynomial $n^3$ backwards
%    having its finite differences?
%\end{question}
%The answer to this question is definitely \textit{Yes}, by utilizing certain interpolation principles.
Interpolation is a process of finding new data points based on the range of a discrete set of known data points.
Interpolation has been well-developed in between 1674--1684
by Issac Newton's fundamental works, nowadays known as foundation of classical interpolation
theory~\cite{meijering2002chronology}.

At that time, in 2016, I was a first-year mechanical engineering undergraduate,
so that due to lack of knowledge and perspective of view I started re-inventing interpolation
formula myself, fueled by purest passion and feeling of mystery.
\textit{All the mathematical laws and relations exist from the very beginning, we only reveal and describe them},
I thought.
That mindset truly inspired me, so my own mathematical journey has been started.
Consider finite differences of cubes $n^3$
\begin{table}[H]
    \begin{center}
        \setlength\extrarowheight{-6pt}
        \begin{tabular}{c|cccc}
            $n$ & $n^3$ & $\Delta(n^3)$ & $\Delta^2(n^3)$ & $\Delta^3(n^3)$ \\
            \hline
            0   & 0     & 1             & 6               & 6               \\
            1   & 1     & 7             & 12              & 6               \\
            2   & 8     & 19            & 18              & 6               \\
            3   & 27    & 37            & 24              & 6               \\
            4   & 64    & 61            & 30              & 6               \\
            5   & 125   & 91            & 36              &                 \\
            6   & 216   & 127           &                 &                 \\
            7   & 343   &               &                 &
        \end{tabular}
    \end{center}
    \caption{Table of finite differences of the polynomial $n^3$.} \label{tab:table}
\end{table}

The problem of interpolation of polynomials is a classical problem in mathematics and has been widely studied in literature.
For instance, Concrete mathematics~\cite{graham1989concrete} gives interpolation of cubes by using
Newton's interpolation formula
\[
    n^3 = 6 \binom{n}{3} + 6 \binom{n}{2} + \binom{n}{1} + 0 \binom{n}{0}
\]
because
\begin{align*}
    f(x) = \Delta^{d} f(0) \binom{x}{d} +  \Delta^{d-1} f(0) \binom{x}{d-1} + \cdots + f(0) \binom{x}{0}
    = \sum_{r=0}^{d} \Delta^{d-r} f(0) \binom{x}{r}
\end{align*}
However, interpolation of cubes can be also done in a different way.
The key point that interpolation formula above iterates over the finite difference of order $d$,
instead it is clear that $n^3$ can be reached as a sum of finite difference $\Delta^1$ of first order
\begin{align*}
    n^3 = \Delta 0^3 + \Delta 1^3 + \Delta 2^3 + \cdots + \Delta (n-1)^3 = \sum_{k=0}^{n-1} \Delta k^3
\end{align*}
We know that $\Delta^3 n^3 = 6$ is the constant for each $n$.
The second difference of cubes $\Delta^2 n^3$ is a linear relation in terms of third order finite difference
$\Delta^3 n^3$.
\begin{align*}
    \Delta^2 n^3 = (n+1) \Delta^3 n^3 = 6(n+1)
\end{align*}
Finally, the first order finite difference $\Delta n^3$ is the following relation in terms of second
order finite difference
\begin{align*}
    \Delta n^3 = \Delta 0^3 + \Delta^2 (n-1)^3 = 1 + \Delta^2 (n-1)^3 = 1 + 6(n-1)
\end{align*}
Meaning that
\begin{align}
    \label{eq:cubes_interpolation}
    \begin{split}
        \Delta(0^3) &= 1+6 \cdot 0 \\
        \Delta(1^3) &= 1+6\cdot0+6\cdot1 \\
        \Delta(2^3) &= 1+6\cdot0+6\cdot1+6\cdot2 \\
        \Delta(3^3) &= 1+6\cdot0+6\cdot1+6\cdot2+6\cdot3
    \end{split}
\end{align}
Finally reaching its generic form
\begin{equation}
    \Delta(n^3) = 1+6\cdot0+6\cdot1+6\cdot2+6\cdot3+\cdots+6\cdot n = 1 + 6 \sum_{k=0}^{n} k
    \label{eq:general-cube-eq}
\end{equation}
Because
\begin{align*}
    \Delta(n^3) = \Delta (n-1)^3 + \Delta^2 (n-1)^3
\end{align*}
Having the relation $n^3 = \Delta 0^3 + \Delta 1^3 + \Delta 2^3 + \cdots + \Delta (n-1)^3$,
we get
\begin{align}
    \label{eq:rearrangement_to_get_cubes}
    n^3 &= [1+6\cdot0]+[1+6\cdot0+6\cdot1]+[1+6\cdot0+6\cdot1+6\cdot2]+\cdots \nonumber \\
    &+[1+6\cdot0+6\cdot1+6\cdot2+\cdots+6\cdot(n-1)]
\end{align}
By rearranging the terms of equation~\eqref{eq:rearrangement_to_get_cubes} it turns into summation
in terms of $k (n-k)$
\begin{equation*}
    \begin{split}
        n^3 &= n + [(n-0) \cdot 6 \cdot 0] + [(n-1)\cdot6\cdot1] + [(n-2)\cdot6\cdot2] + \cdots \\
        &\cdots + [(n-k)\cdot 6 \cdot k] + \cdots + [1\cdot6\cdot(n-1)]
    \end{split}
\end{equation*}
Giving an identity for cubes $n^3$
\begin{equation}
    \label{eq:cube_identity}
    n^3 = \sum_{k=1}^{n} 6k(n-k) + 1
\end{equation}
