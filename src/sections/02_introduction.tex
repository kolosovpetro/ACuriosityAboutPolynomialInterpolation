Back then, in 2016 being a student at the faculty of mechanical engineering,
I remember myself playing with finite differences of the polynomial $n^3$ over the domain of natural numbers $n\in\mathbb{N}$
having at most $0 \leq n \leq 20$ values.
Looking to the values in my finite difference tables, the first and very naive question that came to my mind was
\begin{question}
    Is it possible to re-assemble the value of the polynomial $n^3$ backwards
    having its finite differences?
\end{question}
The answer to this question is definitely \textit{Yes}, by utilizing certain interpolation principles.
Interpolation is a process of finding new data points based on the range of a discrete set of known data points.
Interpolation has been well-developed in between 1674--1684
by Issac Newton's fundamental works, nowadays known as foundation of classical interpolation
theory~\cite{meijering2002chronology}.

At that time, in 2016, I was a first-year mechanical engineering undergraduate,
so that due to lack of knowledge and perspective of view I started re-inventing interpolation
formula myself, fueled by purest passion and feeling of mystery.
\textit{All the mathematical laws and relations exist from the very beginning, we only reveal and describe them},
I thought.
That mindset truly inspired me, so my own mathematical journey has been started.
Let's begin considering the table of finite differences of integer cubes $n^3$
\begin{table}[H]
    \begin{center}
        \setlength\extrarowheight{-6pt}
        \begin{tabular}{c|cccc}
            $n$ & $n^3$ & $\Delta(n^3)$ & $\Delta^2(n^3)$ & $\Delta^3(n^3)$ \\
            \hline
            0   & 0     & 1             & 6               & 6               \\
            1   & 1     & 7             & 12              & 6               \\
            2   & 8     & 19            & 18              & 6               \\
            3   & 27    & 37            & 24              & 6               \\
            4   & 64    & 61            & 30              & 6               \\
            5   & 125   & 91            & 36              &                 \\
            6   & 216   & 127           &                 &                 \\
            7   & 343   &               &                 &
        \end{tabular}
    \end{center}
    \caption{Table of finite differences of the polynomial $n^3$.} \label{tab:table}
\end{table}

